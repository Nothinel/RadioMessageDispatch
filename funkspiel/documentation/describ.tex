\documentclass[12pt,a4paper]{scrreport}
\usepackage[utf8]{inputenc}
\usepackage[ngerman, english]{babel}


\author{Konrad Ritter}
\title{Funkspiel Beschreibung und Regeln}
\begin{document}
\maketitle
\section{Material}
Jeder Spieler benötigt einen Würfel, eine Stoppuhr und einen Stift.
Außerdem sollten alle Spieler digitalen Zugriff auf die für sie erstellten
Dokumente haben. Wenn kein Digitaler Zugriff möglich ist, geht auch ein
Ausdruck. Alle Spieler müssen über Funk oä. verbunden sein. Dabei muss nicht
jeder Spieler direkt jeden anderen kontaktieren können, eine weiterleitung durch
andere Spieler ist durchaus denkbar.

\section{Zusammenfassung}
\begin{itemize}
	\item Alle Spieler spielen gemeinsam gegen das Spiel
	\item Die Gruppe versucht gemeinsam so schnell wie möglich, so viele Punkte
		wie möglich zu sammeln.
	\item Jeder Spieler kann Punkte erwerben, indem er Aufgaben auf seinem
		individuellen Aufgabenblatt löst. Es gibt 3 Typen von Aufgaben.
		\begin{itemize}
			\item Übertragung: Ein Spieler erhält eine Information und muss
				diese an einen bestimmten anderen Spieler übertragen.
			\item Anfrage: Ein Spieler muss Informationen von einem anderen
				Spieler erfragen. 
				%Im passenden Lösungsbogen kann überprüft
				%werden, ob alles richtig übertragen wurde, dort findet sich auch
				%die Punkteanzahl.
			\item Notfall: Der Spieler erhält eine Zahl. Andere Spieler können
				um Hilfe gerufen werden. Jeder um Hilfe gerufene Spieler wirft
				mit seinem Würfel. Die Zahlen aller Helfer werden gesammelt und
				müssen exakt die Zahl des Notfalls ergeben. Hierbei dürfen
				Zahlen addiert und subtrahiert werden. Spieler die Helfen können
				für eine gewisse Zeit keinem anderen Spieler mehr helfen. Je
				größer die Zahl, die Sie beigetragen haben, desto länger die
				Sperre.
		\end{itemize}
\end{itemize}

\end{document}
